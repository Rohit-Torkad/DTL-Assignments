\documentclass[conference]{IEEEtran}
\author{Rohit Bhaurao Torkad}
\date{15/01/2023}
\usepackage{cite}
\usepackage{graphicx}
\usepackage{amsmath}
\usepackage{array}

\title{IEEE research paper}
\author{\IEEEauthorblockN{Rohit Torkad}
	\IEEEauthorblockA{Department of Computer Science\\
		abc@gamil.com}
	\and
	\IEEEauthorblockN{Rushikesh Chavan}
	\IEEEauthorblockA{Department of Mechanical Engineering\\
		xyz.gmail.com}}

\begin{document}
	
	\maketitle
	
	\begin{abstract}
		This document is a model and instructions for
		LATEX. This and the IEEEtran.cls file define the components of
		your paper [title, text, heads, etc.]. *CRITICAL: Do Not Use
		Symbols, Special Characters, Footnotes, or Math in Paper Title
		or Abstract.
	\end{abstract}
	
	\section{Introduction}
	
	This document is a model and instructions for LATEX. Please
	observe the conference page limits.
	
	\section{Background and related work}
	
	The IEEEtran class file is used to format your paper and
	style the text. All margins, column widths, line spaces, and
	text fonts are prescribed; please do not alter them. You may
	note peculiarities. For example, the head margin measures
	proportionately more than is customary. This measurement and
	others are deliberate, using specifications that anticipate your
	paper as one part of the entire proceedings, and not as an
	independent document. Please do not revise any of the current
	designations.
	
	\section{Methodology}

	Before you begin to format your paper, first write and
	save the content as a separate text file. Complete all content
	and organizational editing before formatting. Please note sec-
	tions III-A–III-E below for more information on proofreading,
	spelling and grammar.
	Keep your text and graphic files separate until after the text
	has been formatted and styled. Do not number text heads—
	LATEX will do that for you.
	
	\section{Results}
	
	Define abbreviations and acronyms the first time they are
	used in the text, even after they have been defined in the
	abstract. Abbreviations such as IEEE, SI, MKS, CGS, ac, dc,
	and rms do not have to be defined. Do not use abbreviations
	in the title or heads unless they are unavoidable.
	
	\section{Discussion}
	
	Number equations consecutively. To make your equations
	more compact, you may use the solidus ( / ), the exp
	function, or appropriate exponents. Italicize Roman symbols
	for quantities and variables, but not Greek symbols. Use a
	long dash rather than a hyphen for a minus sign. Punctuate
	equations with commas or periods when they are part of a
	sentence, as in:
	
	\section{Conclusion}
	
	The class file is designed for, but not limited to, six
	authors. A minimum of one author is required for all confer-
	ence articles. Author names should be listed starting from left
	to right and then moving down to the next line. This is the
	author sequence that will be used in future citations and by
	indexing services. Names should not be listed in columns nor
	group by affiliation. Please keep your affiliations as succinct as
	possible (for example, do not


	
\end{document}
